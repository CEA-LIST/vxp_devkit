\chapter{Introduction}

Xvpfloat is a RISC-V instruction-set architecture (ISA) extension to support floating point computation with variable extended precision.
The goals of this extension include:
\vspace{-0.1in}
\begin{itemize}
\parskip 0pt
\itemsep 1pt
\item Separate memory representation and internal representation of floating point numbers.
\item Support IEEE-754 2008 standard~\cite{ieee754-2008} format in memory with support for both extendable and extended formats where total size is a multiple of 8 bits.
\item Internal representation holds automatically computed precision information.
\item Fast switching between multiple memory formats and rounding modes.
\end{itemize}
\vspace{-0.1in}

%\begin{commentary}
%  Commentary on our design decisions is formatted as in this
%  paragraph.  This non-normative text can be skipped if the reader is
%  only interested in the specification itself.
%\end{commentary}

\section{RISC-V Xvpfloat ISA Extension Overview}

Xvpfloat is an extension of RV64I ISA.
It is compatible with other optional RISC-V standard extensions, in particular but not limited to M, A, F and D extensions.

Internal format is fixed and defined in Section~\ref{sec:vpregs}.
However, a special field {\em L} holds information about the current valid portion of the mantissa.
This information can be used in an \implementationdefined\ manner to optimize compute performance.

In-memory format of Xvpfloat extension is IEEE-754 2008 extendable format for floating point numbers.
Mantissa and exponent sizes of encoded floating point numbers can be chosen with arbitratry values in \implementationdefined\ boundaries as defined in Section~\ref{sec:hard_cfg}. 
Appendix~\ref{sec:mem_format} expands briefly on the IEEE-754 2008 standard~\cite{ieee754-2008} format and its support in the Xvpfloat extension.

\section{Memory}

Memory is the same as defined by the RISC-V base specification.

Executing each Xvpfloat machine instruction entails one or more memory accesses, subdivided into {\em implicit} and {\em explicit} accesses.
For each instruction executed, an {\em implicit} memory read (instruction fetch) is done to obtain the encoded instruction to execute.
Load and store defined in this extension can produce multiple memory accesses at machine level due to hardware alignement constraints.

\section{Xvpfloat Instruction Encoding}

Xvpfloat instructions are encoded in the {\em custom-0} major opcode space.
Multiple sub-formats are defined in Figure~\ref{fig:Xvpfloatinstformats} and encode Xvploat instructions.

\begin{figure}[h]
  \begin{center}
  \setlength{\tabcolsep}{4pt}
  \begin{tabular}{@{}p{0.8in}@{}p{0.6in}@{}p{0.8in}@{}p{0.8in}@{}p{0.6in}@{}p{0.8in}@{}p{1in}@{}l}
  \\
  \instbitrange{31}{28} &
  \instbitrange{27}{25} &
  \instbitrange{24}{20} &
  \instbitrange{19}{15} &
  \instbitrange{14}{12} &
  \instbitrange{11}{7} &
  \instbitrange{6}{0} \\
  \cline{1-7}
  \multicolumn{2}{|c|}{Xop7} &
  \multicolumn{1}{c|}{rs2} &
  \multicolumn{1}{c|}{rs1} &
  \multicolumn{1}{c|}{env} &
  \multicolumn{1}{c|}{rd} &
  \multicolumn{1}{c|}{{\em custom-0}} &
  ~Re-type \\
  \cline{1-7}
  \multicolumn{2}{|c|}{Xop7} &
  \multicolumn{1}{c|}{rs2} &
  \multicolumn{1}{c|}{rs1} &
  \multicolumn{1}{c|}{rm} &
  \multicolumn{1}{c|}{rd} &
  \multicolumn{1}{c|}{{\em custom-0}} &
  ~Rr-type \\
  \cline{1-7}
  \multicolumn{2}{|c|}{Xop7} &
  \multicolumn{1}{c|}{rs2} &
  \multicolumn{1}{c|}{rs1} &
  \multicolumn{1}{c|}{type} &
  \multicolumn{1}{c|}{rd} &
  \multicolumn{1}{c|}{{\em custom-0}} &
  ~Rt-type \\
  \cline{1-7}
  \multicolumn{1}{|c|}{Xop4} &
  \multicolumn{1}{c|}{imm[7:5]} &
  \multicolumn{1}{c|}{imm[4:0]} &
  \multicolumn{1}{c|}{rs1} &
  \multicolumn{1}{c|}{env} &
  \multicolumn{1}{c|}{rd} &
  \multicolumn{1}{c|}{{\em custom-0}} &
  ~LI-type \\
  \cline{1-7}
  \multicolumn{1}{|c|}{Xop4} &
  \multicolumn{1}{c|}{imm[7:5]} &
  \multicolumn{1}{c|}{rs2} &
  \multicolumn{1}{c|}{rs1} &
  \multicolumn{1}{c|}{env} &
  \multicolumn{1}{c|}{imm[4:0]} &
  \multicolumn{1}{c|}{{\em custom-0}} &
  ~SI-type \\
  \cline{1-7}
  \end{tabular}
  \end{center}
  \caption{RISC-V Xvpfloat instruction formats, derived from R-type standard format.}
  \label{fig:Xvpfloatinstformats}
  \end{figure}

\section{Exceptions, Traps, and Interrupts}
\label{sec:trap-defn}

RISC-V synchronous traps may be raised when reading from memory or configuring environment registers.
See section~\ref{sec:exc-defn} for definitions of these traps and details on their cause.

\section{UNSPECIFIED Behaviors and Values}

The architecture fully describes what implementations must do and any constraints on what they may do.
In cases where the architecture intentionally does not constrain implementations, the term \unspecified\ is explicitly used.
