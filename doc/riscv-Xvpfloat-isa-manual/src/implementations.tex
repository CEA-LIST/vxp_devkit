\chapter{Xvpfloat implementations}

\section{Rhea implementation}

\subsection{Architecture parameters}

\begin{figure}[ht]
\begin{center}
    \begin{tabular}{|l|c|p{4in}|}
    \hline
    C & 8 & Internal mantissa 64 bits chunks \\
    \hline
    ELEN & 18 & Internal exponent size (bits)\\
    \hline
    \end{tabular}
\end{center}
\caption{RISC-V Xvpfloat Rhea main parameters.}
\label{fig:Xvpfloat_rhea_main_params}
\end{figure}

\begin{figure}[ht]
\begin{center}
    \begin{tabular}{|l|c|p{4in}|}
    \hline
    MLEN & 512 & Internal mantissa size (bits) \\
    \hline
    LLEN & 3 & Internal L field size (bits)\\
    \hline
    HLEN & 26 & {\em P} register header size (bits)\\
    \hline
    VPLEN & 538 & Total size of internal scalar {\em P} register (bits)\\
    \hline
    BIS\textsubscript{MAX} & 531 & Maximum size in memory (bits)\\
    \hline
    BYS\textsubscript{MAX} & 67 & Maximum size in memory (bytes)\\
    \hline
    \end{tabular}
\end{center}
\caption{RISC-V Xvpfloat Rhea derived parameters.}
\label{fig:Xvpfloat_rhea_derived_params}
\end{figure}

\subsection{Memory formats support}

Due to issues that may happen when reading subnormal value with $ES = IEEE\_Esize_{MAX}$ (see Section~\ref{sec:subnormalFormatSupport}), we report strict and partial compliance with IEEE-754 2008 standard.
Partial compliance means that reading a subnormal number $ |x| < 2^{2^{-(ES-1)}}$ may produce incorrect results.
Xvpfloat guarantees that such a number is properly rounded when stored in memory for correct results when reading it back.

\begin{figure}[ht]
\begin{center}
    \begin{tabular}{|l|l|l|}
    \hline
                        & READ                    & WRITE \\
    \hline
                        & $2 \leq Msize \leq 512$ & $2 \leq Msize \leq 512$ \\
    \cline{2-3}
    Strict compliance   & $1 \leq Esize \leq 17$  & $1 \leq Esize \leq 18$  \\
    \cline{2-3}
                        & $1 \leq BYS \leq 66$    & $1 \leq BYS \leq 67$    \\
    \hline
                        & $2 \leq Msize \leq 512$ & $2 \leq Msize \leq 512$ \\
    \cline{2-3}
    Partial compliance  & $1 \leq Esize \leq 18$  & $1 \leq Esize \leq 18$  \\
    \cline{2-3}
                        & $1 \leq BYS \leq 67$    & $1 \leq BYS \leq 67$    \\
    \hline
    \end{tabular}
\end{center}
\caption{RISC-V Xvpfloat Rhea IEEE extendable support.}
\label{fig:Xvpfloat_rhea_ieee_extendable}
\end{figure}

\begin{figure}[ht]
\begin{center}
    \begin{tabular}{|l|l|l|l|l|l|}
    \hline
    Bitsize (k) & ES & MS  & Interchange & Strict compliance & Partial compliance \\
    \hline
    16          & 3  & 12  & X           & X                 &                    \\
    24          & 5  & 18  &             & X                 &                    \\
    32          & 7  & 24  & X           & X                 &                    \\
    40          & 8  & 31  &             & X                 &                    \\
    48          & 9  & 38  &             & X                 &                    \\
    56          & 10 & 45  &             & X                 &                    \\
    64          & 11 & 52  & X           & X                 &                    \\
    72          & 12 & 59  &             & X                 &                    \\
    80          & 12 & 67  &             & X                 &                    \\
    88          & 13 & 74  &             & X                 &                    \\
    96          & 13 & 82  &             & X                 &                    \\
    104         & 14 & 89  &             & X                 &                    \\
    112         & 14 & 97  &             & X                 &                    \\
    120         & 15 & 104 &             & X                 &                    \\
    128         & 15 & 112 & X           & X                 &                    \\
    136         & 15 & 120 &             & X                 &                    \\
    144         & 16 & 127 &             & X                 &                    \\
    152         & 16 & 135 &             & X                 &                    \\
    160         & 16 & 143 & X           & X                 &                    \\
    168         & 17 & 150 &             & X                 &                    \\
    176         & 17 & 158 &             & X                 &                    \\
    184         & 17 & 166 &             & X                 &                    \\
    192         & 17 & 174 & X           & X                 &                    \\
    200         & 18 & 181 &             &                   & X                  \\
    208         & 18 & 189 &             &                   & X                  \\
    216         & 18 & 197 &             &                   & X                  \\
    224         & 18 & 205 & X           &                   & X                  \\
    232         & 18 & 213 &             &                   & X                  \\
    \hline
    \end{tabular}
\end{center}
\caption{RISC-V Xvpfloat Rhea IEEE extended support.}
\label{fig:Xvpfloat_rhea_ieee_extended}
\end{figure}

\subsection{Performance}

\subsubsection{Forewords}

Latencies presented here are measured in isolation with an empty pipeline.
So, it does not account for back-pressure or arbitration conflicts when running other instructions in parallel.

Bandwdith reported here is measured by sending the same instruction repeatedly with the same data and parameters.

Instruction performance is both data-dependent and parameters-dependent.
As a consequence, following sections will include both models and multi-dimensionnal measurements.

\subsubsection{Addition/substraction performance}

\subsubsection{Multiplication performance}

\subsubsection{Conversion performance}

\subsubsection{Load and store performance}

\subsubsection{Move performance}

\subsubsection{Comparison performance}