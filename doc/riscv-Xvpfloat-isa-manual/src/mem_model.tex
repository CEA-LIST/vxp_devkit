\chapter{Memory Model}
\label{sec:mem_model}

As mentioned in previous sections, the Xvpfloat extension provides additional instructions for writing or reading variable precision data in memory. 
The Xvpfloat extensions implements a load-store memory architecture (like the integer "I" base ISA).

Two important properties need to be considered when programming with the Xvpfloat extension.

\begin{enumerate}
    \item A given RISC-V core can issue different kinds of memory instructions (e.g. base ISA ("I") integer load/stores and Xvpfloat load/stores).
    \item Integer load/stores and Xvpfloat load/stores share the same memory.
\end{enumerate}

These properties have implications in the memory coherency and the memory consistency models of the Xvpfloat extension.

\section{Memory Coherency}
\label{sec:mem_coherency}

The memory coherency is defined as the property of the memory subsystem (in a shared-memory system) to provide a consistent view to the different cores accessing the memory. 
This is, when two different cores access a given address, both need to see the same value. 
This can be an issue in systems implementing private data caches on the different cores.

Regarding the memory coherency, for a given core, the Xvpfloat extension assumes that the memory is shared and coherent between the different memory instructions from the different extensions (on a given core). 
In case of multi-core software, as each can implement private cache memories, the memory coherency is not enforced by this ISA and therefore it is IMPLEMENTATION DEFINED.

\section{Memory Consistency}
\label{sec:mem_consistency}

The memory consistency defines the order in which read and write operations (load/stores) are executed.

Regarding the memory consistency, this version of the Xvpfloat extension enforces a serialization for Xvpfloat loads/stores in the same address for a single core. 
For example, when the software issues a Xvpfloat store on an address, and then it issues a Xvpfloat load on the same address, the hardware guarantees that the load will read the last written value (from the same core).

\textbf{This extension does not enforce a serial/sequential order between Xvpfloat memory accesses and the other kinds of memory accesses (e.g. integer) even when these are in the same address. 
If the programmer needs to enforce this order, it needs to use memory fences}.

The RISC-V base ISA provides the data "fence" instruction that ensures that all load/stores issued are completed before executing any new memory instructions. 
This is, if the programmer needs to write the memory using integer memory instructions (respectively Xvpfloats), and then read these data back using Xvpfloat memory instructions (respectively integer), a fence instruction is needed after the former to ensure that the latter read the last written values.

Regarding Xvpfloat load and stores on different addresses from the same core, the order is not specified. 
Therefore, as before, if the software needs a specific order, it needs to issue data "fence" instructions.
